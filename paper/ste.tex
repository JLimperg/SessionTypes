% TODO Clean up contractivity issue. Possible solutions:
% 1. Drop contractivity requirement; handle noncontractive types specially
%    in transitivity proof.
% 2. Thread contractivity assumption through all proofs that require it,
%    notably transitivity.
\documentclass{llncs}
\usepackage[utf8]{inputenc}
\usepackage[T1]{fontenc}
\usepackage{amsmath}
\usepackage{amssymb}
\usepackage{amsfonts}
\usepackage{semantic}

\newcommand*{\Msg}{\mathrm{Msg}}
\newcommand*{\Var}{\mathrm{Var}}
\newcommand*{\Sym}{\mathrm{Sym}}
\newcommand*{\Refl}{\mathrm{Refl}}
\newcommand*{\Trans}{\mathrm{Trans}}
\newcommand*{\Ty}{\mathrm{Ty}}
\newcommand*{\Sy}{\mathfrak{S}}
\newcommand*{\lfp}{\mathrm{lfp}}
\newcommand*{\gfp}{\mathrm{gfp}}
\newcommand*{\cs}{\mathrm{cs}}
\newcommand*{\sequiv}{\sim}
\newcommand*{\union}{\cup}
\newcommand*{\send}{\mathord{!}}
\newcommand*{\recv}{\mathord{?}}
\newcommand*{\echoice}{\oplus}
\newcommand*{\ichoice}{\mathop{\&}}
\newcommand*{\concat}{\cdot}
\newcommand*{\wf}[1]{\text{$#1$ {\normalfont \rmfamily wellformed}}}
\newcommand*{\contractive}[1]{\text{$#1$ {\normalfont \rmfamily contractive}}}
\newcommand*{\envimpl}       [2]{#1 \vdash #2}
\newcommand*{\envimplchecked}[2]{\envimpl{#1}{\text{$#2$ {\normalfont \rmfamily checked}}}}
\newcommand*{\envimplok}     [2]{\envimpl{#1}{\text{$#2$ {\normalfont \rmfamily ok}}}}
\renewcommand*{\|}{\;|\;}
\renewcommand*{\epsilon}{\varepsilon}
\newcommand*{\eqwith}[1]{\mathrel{\overset{\makebox[0pt]{\mbox{\normalfont\tiny\rmfamily #1}}}{=}}}

\begin{document}
\title{Trace Equivalence for Session Types}
\author{Jannis Limperg}
\institute{Freiburg University}
\maketitle

\begin{definition}[Session Types]
  \label{def:session_types}
  \begin{eqnarray*}
    F_{\Ty}(\tau)
      &:=&     \{1\} \\
      &\union& \{\send B.S \| S \in \tau, B \in \Msg\} \\
      &\union& \{\recv B.S \| S \in \tau, B \in \Msg\} \\
      &\union& \{S_1 \echoice S_2 \| S_1, S_2 \in \tau\} \\
      &\union& \{S_1 \ichoice S_2 \| S_1, S_2 \in \tau\} \\
      &\union& \{\mu X.S \| X \in \Var, S \in \tau\} \\
    \Ty &:=& \lfp(F_{\Ty}).
  \end{eqnarray*}
  where $\Msg$ is a base set of messages, $\Var$ is a base set of variable
  names, and both are distinct from $\Ty$ and each other. Additionally,
  $\{0, 1\} \cap \Msg = \emptyset$.

  In the following, we use $B$ for messages and $X, Y, \dots$ for variables and
  occasionally omit the explicit quantification.
\end{definition}

\begin{definition}[Contractivity]
  \label{def:contractivity}
  A session type $S$ is contractive iff no part of $S$ is of the form
  \begin{equation*}
    \mu X_1.\dots.\mu X_n.X_1.
  \end{equation*}
  The set of contractive session types is
  \begin{equation*}
    \Sy := \{S \| S \in \Ty, \contractive{S}\}.
  \end{equation*}
\end{definition}

\begin{definition}[Trace Language]
  \label{def:trace_language}
  \begin{eqnarray}
    \label{def:trace_language:1}
    L_\eta(1)    &:=& \{ \epsilon \} \\
    \label{def:trace_language:send}
    L_\eta(\send B.S) &:=& \{ \send B \} \concat L_\eta(S) \quad \forall S \in \Sy \\
    \label{def:trace_language:recv}
    L_\eta(\recv B.S) &:=& \{ \recv B \} \concat L_\eta(S) \quad \forall S \in \Sy \\
    \label{def:trace_language:echoice}
    L_\eta(S_1 \echoice S_2)
      &:=&   \{ \send 0 \} \concat L_\eta(S_1)
      \union \{ \send 1 \} \concat L_\eta(S_2)
      \quad \forall S_1, S_2 \in \Sy \\
    \label{def:trace_language:ichoice}
    L_\eta(S_1 \ichoice S_2)
      &:=&   \{ \recv 0 \} \concat L_\eta(S_1)
      \union \{ \recv 1 \} \concat L_\eta(S_2)
      \quad \forall S_1, S_2 \in \Sy \\
    \label{def:trace_language:mu}
    L_\eta(\mu X.S) &:=& \gfp(LX \mapsto L_{\eta[X \mapsto LX]}(S))
      \quad \forall S \in \Sy \\
    \label{def:trace_language:var}
    L_\eta(X) &:=& \eta(X)
  \end{eqnarray}
  where $\eta\colon \Var \to \Sigma^\infty$.
\end{definition}

\begin{definition}[Wellformed]
  \label{def:wellformed}
  Let the predicates $\mathrm{ok}$ and $\mathrm{checked}$ be defined
  by the following inductive inference system:

  \inference{}{\envimplok{XS}{1}}

  \bigskip

  \inference{\envimplchecked{XS}{S}}
            {\envimplok{XS}{\send B.S}}

  \bigskip

  \inference{\envimplchecked{XS}{S}}
            {\envimplok{XS}{\recv B.S}}

  \bigskip

  \inference{\envimplchecked{XS}{S_1} &
             \envimplchecked{XS}{S_2}}
            {\envimplok{XS}{S_1 \echoice S_2}}

  \bigskip

  \inference{\envimplchecked{XS}{S_1} &
             \envimplchecked{XS}{S_2}}
            {\envimplok{XS}{S_1 \ichoice S_2}}

  \bigskip

  \inference{\envimplok{XS, X}{S}}
            {\envimplok{XS}{\mu X.S}}

  \bigskip

  \inference{}{\envimplchecked{XS, X}{X}}

  \bigskip

  \inference{\envimplok{XS}{S}}
            {\envimplchecked{XS}{S}}

  \begin{eqnarray*}
    &&     \wf{S \in \Ty} \\
    &\iff& \envimplok{0}{S}
  \end{eqnarray*}
\end{definition}

\begin{theorem}[Structure of Trace Languages]
  \label{th:struct_trace_langs}
  \begin{eqnarray*}
    &&         \forall S \in \Ty, \eta. \\
    &&         \wf{S} \\
    &\implies& L_\eta(S) = \{\epsilon\} \\
    &\lor&     L_\eta(S) \subseteq \{\send B\} \concat \Sigma^\infty \\
    &\lor&     L_\eta(S) \subseteq \{\recv B\} \concat \Sigma^\infty \\
    &\lor&     L_\eta(S) \subseteq (\{\send 0\} \concat \Sigma^\infty) \union (\{\send 1\} \concat \Sigma^\infty) \\
    &\lor&     L_\eta(S) \subseteq (\{\recv 0\} \concat \Sigma^\infty) \union (\{\recv 1\} \concat \Sigma^\infty)
  \end{eqnarray*}
\end{theorem}

\begin{proof}
  Simplified induction principle for derivation trees of well-formedness
  (omitting assumptions not required for the following proof):

  \begin{eqnarray*}
    &&      \forall P. \\
    &&      P(1) \\
    &\land& (\forall S.\, P(S) \implies P(\send B.S)) \\
    &\land& (\forall S.\, P(S) \implies P(\recv B.S)) \\
    &\land& (\forall S_1, S_2.\, P(S_1) \land P(S_2) \implies P(S_1 \echoice S_2)) \\
    &\land& (\forall S_1, S_2.\, P(S_1) \land P(S_2) \implies P(S_1 \ichoice   S_2)) \\
    &\land& (\forall S, X.\, P(S) \implies P(\mu X.S)) \\
    &\implies& (\forall S.\, P(S))
  \end{eqnarray*}

  Induction according to the above induction principle with
  \begin{equation*}
    P(S) := \forall \eta.\, L_\eta(S) = \{\epsilon\} \lor L_\eta(S) \subseteq
            \{\send B\} \concat \Sigma^\infty \lor \dots
  \end{equation*}

  \begin{enumerate}
    \item
      \label{case:struct_trace_langs:1}
      $P(1)$.

      By def.,
      \begin{equation*}
        \forall \eta.\, L_\eta(1) = \{\epsilon\}.
      \end{equation*}

    \item
      \label{case:struct_trace_langs:send}
      $\forall S.\, P(S) \implies P(\send B.S)$.

      Choose $S, \eta$. By def.,
      \begin{equation*}
        L_\eta(\send B.S) = \{\send B\} \concat L_\eta(S).
      \end{equation*}
      Due to $P(S)$,
      \begin{equation*}
        L_\eta(S) \subseteq \Sigma^\infty.
      \end{equation*}
      Thus
      \begin{equation*}
        L_\eta(\send B.S) \subseteq \{\send B\} \concat \Sigma^\infty.
      \end{equation*}

    \item
      \label{case:struct_trace_langs:recv}
      $\forall S.\, P(S) \implies P(\recv B.S)$.

      Analogous to case \ref{case:struct_trace_langs:send}.

    \item
      \label{case:struct_trace_langs:echoice}
      $\forall S_1, S_2.\, P(S_1) \land P(S_2) \implies P(S_1 \echoice S_2)$.

      Choose $S_1, S_2, \eta$. By def.,
      \begin{equation*}
        L_\eta(S_1 \echoice S_2) = (\{\send 0\} \concat L_\eta(S_1)) \union (\{\send 1\} \concat L_\eta(S_2)).
      \end{equation*}
      Due to $P(S_1)$,
      \begin{equation*}
        L_\eta(S_1) \subseteq \Sigma^\infty.
      \end{equation*}
      Due to $P(S_2)$,
      \begin{equation*}
        L_\eta(S_2) \subseteq \Sigma^\infty.
      \end{equation*}
      Thus
      \begin{equation*}
        L_\eta(S_1 \echoice S_2) \subseteq (\{\send 0\} \concat \Sigma^\infty) \union (\{\send 1\} \concat \Sigma^\infty).
      \end{equation*}

    \item
      \label{case:struct_trace_langs:ichoice}
      $\forall S_1, S_2.\, P(S_1) \land P(S_2) \implies P(S_1 \ichoice S_2)$.

      Analogous to case \ref{case:struct_trace_langs:echoice}.

    \item
      \label{case:struc_trace_langs:mu}
      $\forall S, X.\, P(S) \implies P(\mu X.S)$.

      Choose $S, X, \eta$.
      Let
      \begin{equation*}
        L := \gfp(LX \mapsto L_{\eta[X \mapsto LX]}(S)).
      \end{equation*}
      Since $L$ is a fixed point,
      \begin{equation}
        \label{eq:struct_trace_langs:fp}
        L = L_{\eta[X \mapsto L]}(S).
      \end{equation}
      By def.,
      \begin{eqnarray*}
        L_\eta(\mu X.S)
          &=& L \\
          &\eqwith{(\ref{eq:struct_trace_langs:fp})}& L_{\eta[X \mapsto L]}(S).
      \end{eqnarray*}
      Due to $P(S)$,
      \begin{equation*}
        L_\eta(\mu X.S) = \{\epsilon\} \lor L_\eta(\mu X.S) \subseteq \dots.
      \end{equation*}
  \end{enumerate}
\end{proof}

\begin{definition}[Session Type Equivalence]
  \label{def:equivalence}
  \begin{eqnarray}
    F_\sequiv(\sequiv)
    &:=&     \label{def:equivalence:mu1}
             \{(\mu X.S, S') \| (S[X \mapsto \mu X.S], S') \in \sequiv\} \\
    &\union& \label{def:equivalence:mu2}
             \{(S, \mu X.S') \| (S, S'[X \mapsto \mu X.S']) \in \sequiv\} \\
    &\union& \label{def:equivalence:ichoice}
             \{(S_1 \ichoice S_2, S_1' \ichoice S_2') \| (S_1, S_1') \in \sequiv \land (S_2, S_2') \in \sequiv\} \\
    &\union& \label{def:equivalence:echoice}
             \{(S_1 \echoice S_2, S_1' \echoice S_2') \| (S_1, S_1') \in \sequiv \land (S_2, S_2') \in \sequiv\} \\
    &\union& \label{def:equivalence:send}
             \{(\send B.S, \send B.S') \| (S, S') \in \sequiv\} \\
    &\union& \label{def:equivalence:recv}
             \{(\recv B.S, \recv B.S') \| (S, S') \in \sequiv\} \\
    &\union& \label{def:equivalence:1}
             \{(1,1)\}
  \end{eqnarray}

  \begin{equation*}
    \sequiv := \gfp(F_\sequiv).
  \end{equation*}
\end{definition}

\begin{lemma}[Monotonicity of $F_\sequiv$]
  \label{th:monotonicity}
  \begin{equation*}
    S \subseteq T \implies F_\sequiv(S) \subseteq F_\sequiv(T).
  \end{equation*}
\end{lemma}

\begin{proof}
  Trivial. %TODO
\end{proof}

\begin{definition}[Symmetric Closure]
  \label{def:symmetric_closure}
  \begin{equation*}
    \Sym(R) := \{ (x_2, x_1) \| (x_1, x_2) \in R \}.
  \end{equation*}
\end{definition}

\begin{theorem}[Symmetry of $\sequiv$]
  \label{th:symmetry}
  \begin{equation*}
    \Sym(\sequiv) \subseteq \sequiv.
  \end{equation*}
\end{theorem}

\begin{proof}
  We will prove that
  \begin{equation*}
    \Sym(\sequiv) \subseteq F_\sequiv(\Sym(\sequiv)).
  \end{equation*}
  which, by the fixed-point theorem, implies that
  \begin{equation*}
    \Sym(\sequiv) \subseteq \sequiv.
  \end{equation*}

  Choose $(t, t') \in \Sym(\sequiv)$. By def.,
  \begin{equation*}
    (t', t) \in \sequiv.
  \end{equation*}

  Case analysis on the derivation of $(t', t) \in \sequiv$
  (Def. \ref{def:equivalence}).

  \begin{enumerate}
    \item
      \label{case:symmetry:mu1}
      Derivation by Eq. \ref{def:equivalence:mu1}, i.e.
      \begin{eqnarray*}
        &&      (t', t) = (\mu X.S, S') \\
        &\land& (S[X \mapsto \mu X.S], S') \in \sequiv
      \end{eqnarray*}
      Thus, by Def. \ref{def:symmetric_closure},
      \begin{equation*}
        (S', S[X \mapsto \mu X.S]) \in \Sym(\sequiv).
      \end{equation*}
      Thus, by Eq. \ref{def:equivalence:mu1},
      \begin{equation*}
        (t, t') = (S', \mu X.S, S') \in F_\sequiv(\Sym(\sequiv)).
      \end{equation*}

    \item
      \label{case:symmetry:mu2}
      Derivation by Eq. \ref{def:equivalence:mu2}. Analogous to case
      \ref{case:symmetry:mu1}.

    \item
      \label{case:symmetry:ichoice}
      Derivation by Eq. \ref{def:equivalence:ichoice}, i.e.
      \begin{eqnarray*}
        && (t', t) = (S_1' \ichoice S_2', S_1 \ichoice S_2) \\
        &\land& (S_1', S_1) \in \sequiv \\
        &\land& (S_2', S_2) \in \sequiv.
      \end{eqnarray*}
      Thus, by Def. \ref{def:symmetric_closure},
      \begin{eqnarray*}
        &&      (S_1, S_1') \in \Sym(\sequiv) \\
        &\land& (S_2, S_2') \in \Sym(\sequiv).
      \end{eqnarray*}
      Thus, by Eq. \ref{def:equivalence:ichoice},
      \begin{equation*}
        (t, t') = (S_1 \ichoice S_2, S_1' \ichoice S_2') \in F_\sequiv(\Sym(\sequiv)).
      \end{equation*}

    \item
      \label{case:symmetry:echoice}
      Derivation by Eq. \ref{def:equivalence:echoice}. Analogous to case
      \ref{case:symmetry:ichoice}.

    \item
      \label{case:symmetry:send}
      Derivation by Eq. \ref{def:equivalence:send}, i.e.
      \begin{eqnarray*}
        &&      (t', t) = (\send B.S', \send B.S) \\
        &\land& (S', S) \in \sequiv.
      \end{eqnarray*}
      Thus, by Def. \ref{def:symmetric_closure},
      \begin{equation*}
        (S, S') \in \Sym(\sequiv).
      \end{equation*}
      Thus, by Eq. \ref{def:equivalence:send},
      \begin{equation*}
        (t, t') = (\send B.S, \send B.S') \in F_\sequiv(\Sym(\sequiv)).
      \end{equation*}

    \item
      \label{case:symmetry:recv}
      Derivation by Eq. \ref{def:equivalence:recv}. Analogous to case
      \ref{case:symmetry:send}.

    \item
      \label{case:symmetry:1}
      Derivation by Eq. \ref{def:equivalence:1}, i.e.
      \begin{equation*}
        (t', t) = (1, 1).
      \end{equation*}
      By Eq. \ref{def:equivalence:1},
      \begin{equation*}
        (t, t') = (1, 1) \in F_\sequiv(\Sym(\sequiv)).
      \end{equation*}
  \end{enumerate}
\end{proof}

\begin{definition}[Reflexive Closure]
  \label{def:reflexive_closure}
  \begin{equation*}
    \Refl(S) := \{ (x, x) \| x \in S \}.
  \end{equation*}
\end{definition}

\begin{theorem}[Reflexivity of $\sequiv$]
  \label{th:reflexivity}
  \begin{equation*}
    \Refl(\Ty) \subseteq \sequiv.
  \end{equation*}
\end{theorem}

\begin{proof}
  Let
  \begin{equation*}
    R := \Refl(\Ty) \union \{ (\mu X.S, S[X \mapsto \mu X.S]) \| S \in \Ty \}.
  \end{equation*}
  We will show that
  \begin{equation*}
    R \subseteq F_\sequiv(R)
  \end{equation*}
  which implies, by the fixed point theorem,
  \begin{equation*}
    \Refl(\Ty) \subseteq R \subseteq \sequiv.
  \end{equation*}

  Choose $t \in \Ty$. Case analysis on $t$.

  \begin{enumerate}
    \item
      \label{case:reflexivity:1}
      $t = 1$. By Eq. \ref{def:equivalence:1}, $(1,1) \in F_\sequiv(R)$.

    \item
      \label{case:reflexivity:send}
      $t = \send B.S$.

      By def.,
      \begin{equation*}
        (S, S) \in R.
      \end{equation*}
      Thus, by Eq. \ref{def:equivalence:send},
      \begin{equation*}
        (\send B.S, \send B.S) \in F_\sequiv(R).
      \end{equation*}

    \item
      \label{case:reflexivity:recv}
      $t = \recv B.S$.

      Analogous to case \ref{case:reflexivity:send}.

    \item
      \label{case:reflexivity:echoice}
      $t = S_1 \echoice S_2$.

      By def.,
      \begin{eqnarray*}
        (S_1, S_1) &\in& R \\
        (S_2, S_2) &\in& R.
      \end{eqnarray*}
      Thus, by Eq. \ref{def:equivalence:echoice},
      \begin{equation*}
        (S_1 \echoice S_2, S_1 \echoice S_2) \in F_\sequiv(R).
      \end{equation*}

    \item
      \label{case:reflexivity:ichoice}
      $t = S_1 \ichoice S_2$.

      Analogous to case \ref{case:reflexivity:echoice}.

    \item
      \label{case:reflexivity:mu}
      $t = \mu X.S$.

      By def.,
      \begin{equation*}
        (\mu X.S, S[X \mapsto \mu X.S]) \in R.
      \end{equation*}
      Thus, by Eq. \ref{def:equivalence:mu2},
      \begin{equation*}
        (\mu X.S, \mu X.S) \in F_\sequiv(R).
      \end{equation*}
  \end{enumerate}
\end{proof}

\begin{definition}[Complete Substitution]
  \label{def:complete_substitution}
  \begin{eqnarray*}
    \cs          &\colon& \Ty \to \Ty \\
    \cs(\mu X.S) &:=&     \cs(S[X \mapsto \mu X.S]) \\
    \cs(t)       &:=&     t \quad \nexists X, S.\, t = \mu X.S
  \end{eqnarray*}
\end{definition}

\begin{lemma}[Substitution Preserves Shape]
  \label{lemma:substitution_shape}
  \begin{eqnarray*}
    &&      \forall S, S_1, S_2, X, T.\, \\
    &&      \exists S', S_1', S_2'. \\
    &&      (\send B.S)[X \mapsto T] = \send B.S' \\
    &\land& (\recv B.S)[X \mapsto T] = \recv B.S' \\
    &\land& (S_1 \echoice S_2)[X \mapsto T] = S_1' \echoice S_2' \\
    &\land& (S_1 \ichoice S_2)[X \mapsto T] = S_1' \ichoice S_2' \\
    &\land& 1[X \mapsto T] = 1 \\
    &\land& (Y \neq X \implies Y[X \mapsto T] = Y)
  \end{eqnarray*}
\end{lemma}

\begin{proof}
  By def. of subst.
\end{proof}

\begin{lemma}[Complete Substitution Preserves Shape]
  \label{lemma:complete_substitution_shape}
  \begin{eqnarray*}
    &&         \forall t := \mu X_1.\dots.\mu X_n.S \in \Ty, B, T, T_1, T_2, X. \\
    &&         S \neq X_1 \\
    &\implies& (S = \send B.T \implies (\exists T'.\, \cs(t) = \send B.T')) \\
    &\land&    (S = \recv B.T \implies (\exists T'.\, \cs(t) = \recv B.T')) \\
    &\land&    (S = T_1 \echoice T_2 \implies (\exists T_1', T_2'.\, \cs(t) = T_1' \echoice T_2')) \\
    &\land&    (S = T_1 \ichoice T_2 \implies (\exists T_1', T_2'.\, \cs(t) = T_1' \ichoice T_2')) \\
    &\land&    (S = X \implies \cs(t) = X)
  \end{eqnarray*}
\end{lemma}

\begin{proof}
  Choose $t, B, T, T_1, T_2, X$. Induction over $n$:
  \begin{enumerate}
    \item $n = 0$, i.e. $t = S$, $\cs(t) = \cs(S)$.

      By Def. \ref{def:complete_substitution},
      \begin{enumerate}
        \item $S = \send B.T \implies \cs(t) = S$.
        \item $S = \recv B.T \implies \cs(t) = S$.
        \item $S = T_1 \echoice T_2 \implies \cs(t) = S$.
        \item $S = T_1 \ichoice T_2 \implies \cs(t) = S$.
        \item $S = X \implies \cs(t) = S$.
      \end{enumerate}

    \item $n \to n + 1$, i.e. $t = \mu X_1.\dots.\mu X_{n+1}.S$.

      By Def. \ref{def:complete_substitution},
      \begin{equation*}
        \cs(t) = \cs(\mu X_2.\dots.\mu X_{n+1}.(S[X_1 \mapsto t])).
      \end{equation*}

      For $S = \send B.T$, $S = \recv B.T$, $S = T_1 \echoice T_2$ and
      $S = T_1 \ichoice T_2$, the goal follows directly from Lemma
      \ref{lemma:substitution_shape} and the induction hypothesis.

      For $S = X$: By assumption,
      \begin{equation*}
        X \neq X_1.
      \end{equation*}
      Thus, by Lemma \ref{lemma:substitution_shape},
      \begin{equation*}
        S[X_1 \mapsto t] = X[X_1 \mapsto t] = X.
      \end{equation*}
      Thus, the goal follows directly from the induction hypothesis.
  \end{enumerate}
\end{proof}

\begin{lemma}[Variables in $\sequiv$]
  \label{lemma:variables_equivalence}
  \begin{eqnarray*}
    &&     \forall X \in \Var.\, \nexists t \in \Sy. \\
    &&     (X, t) \in \sequiv \\
    &\lor& (t, X) \in \sequiv.
  \end{eqnarray*}
\end{lemma}

\begin{proof}
  Choose $X$.

  \begin{enumerate}
    \item
      \label{case:variables_equivalence:1}
      $\nexists t \in \Sy.\, (X, t) \in \sequiv$.

      For contradiction, choose $t \in \Sy$ such that
      \begin{equation*}
        (X, t) \in \sequiv.
      \end{equation*}
      By Lemma \ref{lemma:inversion_longest_muprefixes:one-sided},
      \begin{equation*}
        (X, \cs(t)) \in \sequiv.
      \end{equation*}
      By Lemma \ref{lemma:complete_substitution_shape},
      \begin{eqnarray*}
        &&     \exists B, T, T_1, T_2, X. \\
        &&     \cs(t) = \send B.T \\
        &\lor& \cs(t) = \recv B.T \\
        &\lor& \cs(t) = T_1 \echoice T_2 \\
        &\lor& \cs(t) = T_1 \ichoice T_2 \\
        &\lor& \cs(t) = 1 \\
        &\lor& \cs(t) = X
      \end{eqnarray*}
      Thus, inversion of Def. \ref{def:equivalence} leads to a contradiction.

    \item
      \label{case:variables_equivalence:2}
      $\nexists t \in \Sy.\, (t, X) \in \sequiv$.

      By Theorem \ref{th:symmetry}, $(t, X) \in \sequiv$ would imply
      $(X, t) \in \sequiv$. By case \ref{case:variables_equivalence:1}, this
      leads to a contradiction.
  \end{enumerate}
\end{proof}

\begin{lemma}[Inversion of Longest $\mu$-Prefixes]
  \label{lemma:inversion_longest_muprefixes}
  \begin{eqnarray*}
    && \forall t, t' \in \Ty. \\
    && (t, t') \in \sequiv \\
    &\implies& (\cs(t), \cs(t')) \in \sequiv
  \end{eqnarray*}
\end{lemma}

\begin{proof}
  Choose
  \begin{eqnarray*}
    t  &:=& \mu X_1.\dots.\mu X_n.S \\
    t' &:=& \mu Y_1.\dots.\mu Y_m.T
  \end{eqnarray*}
  with $n \geq 0$, $m \geq 0$.

  Induction over $n + m$:
  \begin{enumerate}
    \item $n + m = 0$, i.e. $n = m = 0$. By Def. \ref{def:complete_substitution},
      $\cs(t) = t$ and $\cs(t') = t'$, thus by assumption
      $(\cs(t), \cs(t')) \in \sequiv$.

    \item $n + m \to n + m + 1$.

      Induction hypothesis:
      \begin{eqnarray*}
        && \forall t = \mu X_1.\dots.\mu X_{n'}.S, t' =\mu Y_1.\dots.\mu Y_{m'}.T. \\
        && n' + m' < n + m \\
        &\implies& (t, t') \in \sequiv \\
        &\implies& (\cs(t), \cs(t')) \in \sequiv.
      \end{eqnarray*}

      Case analysis on the derivation of $(t, t') \in \sequiv$:
      \begin{enumerate}
        \item
          \label{case:muprefixes:mu1}
          Derivation by Eq. \ref{def:equivalence:mu1}, thus
          \begin{equation*}
            ((\mu X_2.\dots\mu X_n.S)[X_1 \mapsto \mu X_1.\dots.\mu X_n.S], t') \in \sequiv.
          \end{equation*}
          Thus, by the definition of substitution,
          \begin{equation*}
            (\mu X_2.\dots\mu X_n.(S[X_1 \mapsto \mu X_1.\dots.\mu X_n.S]), t') \in \sequiv.
          \end{equation*}
          Thus, by the induction hypothesis and Def.
          \ref{def:complete_substitution},
          \begin{eqnarray*}
            &&  (\cs(\mu X_2.\dots\mu X_n.(S[X_1 \mapsto \mu X_1.\dots.\mu X_n.S])), \cs(t')) \\
            &=& (S[X_1 \mapsto \mu X_1.\dots.\mu X_n.S]\dots[X_n \mapsto \mu X_n.S], \cs(t')) \\
            &=& (\cs(t), \cs(t')) \\
            &\in& \sequiv.
          \end{eqnarray*}

        \item
          \label{case:muprefixes:mu2}
          Derivation by Eq. \ref{def:equivalence:mu2}: symmetric to case
          \ref{case:muprefixes:mu1}.
      \end{enumerate}
  \end{enumerate}
\end{proof}

\begin{lemma}[One-Sided Inversion of Longest $\mu$-Prefix]
  \label{lemma:inversion_longest_muprefixes:one-sided}
  \begin{eqnarray*}
    && \forall t, t' \in \Ty. \\
    && (t, t') \in \sequiv \\
    &\implies& (\cs(t), t') \in \sequiv \\
    &\land&    (t, \cs(t')) \in \sequiv
  \end{eqnarray*}
\end{lemma}

\begin{proof}
  % TODO A more formal proof would require induction again.
  Choose
  \begin{eqnarray*}
    t  &:=& \mu X_1.\dots.\mu X_n.S \\
    t' &:=& \mu Y_1.\dots.\mu Y_m.T
  \end{eqnarray*}
  By Lemma \ref{lemma:inversion_longest_muprefixes},
  \begin{equation*}
    (\cs(t), \cs(t')) \in \sequiv.
  \end{equation*}
  By Eq. \ref{def:equivalence:mu1} applied $n$ times,
  \begin{equation*}
    (t, \cs(t')) \in \sequiv.
  \end{equation*}
  By Eq. \ref{def:equivalence:mu2} applied $m$ times,
  \begin{equation*}
    (\cs(t), t') \in \sequiv.
  \end{equation*}
\end{proof}

\begin{lemma}[Structure of Complete Substitution]
  \label{lemma:complete_substitution_shape:equivalence}
  \begin{eqnarray*}
    && \forall t \in \Sy.\, \exists T, T_1, T_2 \in \Sy. \\
    && (\exists t'.\, (t, t') \in \sequiv) \\
    &\implies& \cs(t) = 1 \\
    &\lor&     \cs(t) = T_1 \echoice T_2 \\
    &\lor&     \cs(t) = T_1 \ichoice T_2 \\
    &\lor&     \cs(t) = \send B.T \\
    &\lor&     \cs(t) = \recv B.T
  \end{eqnarray*}
\end{lemma}

\begin{proof}
  Choose $t \in \Sy$.
  By Lemma \ref{lemma:complete_substitution_shape},
  \begin{eqnarray*}
    &&     \exists T, T_1, T_2, X. \\
    &&     \cs(t) = \send B.T \\
    &\lor& \cs(t) = \recv B.T \\
    &\lor& \cs(t) = T_1 \echoice T_2 \\
    &\lor& \cs(t) = T_1 \ichoice T_2 \\
    &\lor& \cs(t) = X \\
  \end{eqnarray*}

  Assume for contradiction that $\cs(t) = X$. By assumption,
  \begin{equation*}
    \exists t'.\, (t, t') \in \sequiv.
  \end{equation*}
  Thus, by Lemma \ref{lemma:inversion_longest_muprefixes:one-sided},
  \begin{equation*}
    (X, t') = (\cs(t), t') \in \sequiv.
  \end{equation*}
  By Lemma \ref{lemma:variables_equivalence}, this leads to a contradiction
\end{proof}


\begin{definition}[Transitive Closure]
  \label{def:transitive_closure}
  \begin{equation*}
    \Trans(R) := \{ (x_1, x_3) \| (x_1, x_2) \in R \land (x_2, x_3) \in R \}.
  \end{equation*}
\end{definition}

\begin{theorem}[Transitivity of $\sequiv$]
  \label{th:transitivity}
  \begin{equation*}
    \Trans(\sequiv) \subseteq \sequiv.
  \end{equation*}
\end{theorem}

\begin{proof}
  We will show that
  \begin{equation*}
    \Trans(\sequiv) \subseteq F_\sequiv(\Trans(\sequiv))
  \end{equation*}
  which, by the fixed-point theorem, implies
  \begin{equation*}
    \Trans(\sequiv) \subseteq \sequiv.
  \end{equation*}

  Choose $t_1, t_3$ with $(t_1, t_3) \in \Trans(\sequiv)$. 
  By def.,
  \begin{equation*}
    \exists t_2.\, (t_1, t_2) \in \sequiv \land (t_2, t_3) \in \sequiv.
  \end{equation*}

  Case analysis on the derivation of $(t_1, t_2) \in \sequiv$:
  \begin{enumerate}
    \item
      \label{case:transitivity:mu1}
      $(t_1, t_2) = (\mu X.S, t_2)$; $(S[X \mapsto \mu X.S], t_2) \in \sequiv$.

      By def.,
      \begin{equation*}
        (S[X \mapsto \mu X.S], t_3) \in \Trans(\sequiv).
      \end{equation*}
      Thus, by Eq. \ref{def:equivalence:mu1},
      \begin{equation*}
        (t_1, t_3) = (\mu X.S, t_3) \in F_\sequiv(\Trans(\sequiv)).
      \end{equation*}

    \item
      \label{case:transitivity:mu2}
      $(t_1, t_2) = (t_1, \mu X.S)$; $(t_1, S[X \mapsto \mu X.S]) \in \sequiv$.

      By assumption,
      \begin{eqnarray*}
        &&      (t_1, \mu X.S) \in \sequiv \\
        &\land& (\mu X.S, t_3) \in \sequiv
      \end{eqnarray*}

      By Lemma \ref{lemma:inversion_longest_muprefixes:one-sided},
      \begin{eqnarray*}
        && (t_1, \cs(t_2)) \in \sequiv \\
        && (\cs(t_2), t_3) \in \sequiv
      \end{eqnarray*}

      Inversion on the derivation of $(t_1, \cs(t_2)) \in \sequiv$.
      \begin{enumerate}
        \item Derivation by Eq. \ref{def:equivalence:mu1}
          ($t_1 = \mu Y.U$).
          See case \ref{case:transitivity:mu1}.
        \item Derivation by Eq. \ref{def:equivalence:mu2}
          ($\cs(t_2) = \mu Y.U$).
          Impossible due to Lemma \ref{lemma:complete_substitution_shape:equivalence}.
        \item Derivation by Eq. \ref{def:equivalence:send}
          ($t_1 = \send B.U$, $\cs(t_2) = \send B.V$).
          See case \ref{case:transitivity:send}.
        \item Derivation by Eq. \ref{def:equivalence:recv}
          ($t_1 = \recv B.U$, $\cs(t_2) = \recv B.V$).
          See case \ref{case:transitivity:recv}.
        \item Derivation by Eq. \ref{def:equivalence:echoice}
          ($t_1 = U \echoice U'$, $\cs(t_2) = V \echoice V'$).
          See case \ref{case:transitivity:echoice}.
        \item Derivation by Eq. \ref{def:equivalence:ichoice}
          ($t_1 = U \ichoice U'$, $\cs(t_2) = V \ichoice V'$).
          See case \ref{case:transitivity:ichoice}.
        \item Derivation by Eq. \ref{def:equivalence:1}
          ($t_1 = 1$, $\cs(t_2) = 1$).
          See case \ref{case:transitivity:1}.
      \end{enumerate}

    \item
      \label{case:transitivity:send}
      $(t_1, t_2) = (\send B.S, \send B.S')$; $(S, S') \in \sequiv$.

      Case analysis on the derivation of
      $(t_2, t_3) = (\send B.S', t_3) \in \sequiv$:
      \begin{enumerate}
        \item $t_3 = \send B.S''$.

          By inversion,
          \begin{eqnarray*}
            && (S, S') \in \sequiv \\
            && (S', S'') \in \sequiv
          \end{eqnarray*}
          Thus, by def.,
          \begin{equation*}
            (S, S'') \in \Trans(\sequiv).
          \end{equation*}
          Thus, by Eq. \ref{def:equivalence:send},
          \begin{equation*}
            (t_1, t_3) = (\send B.S, \send B.S'') \in F_\sequiv(\Trans(\sequiv)).
          \end{equation*}

        \item
          \label{case:transitivity:send:mu2}
           $t_3 = \mu X.S''$.

          By assumption,
          \begin{eqnarray*}
            && (\send B.S, \send B.S') \in \sequiv \\
            && (\send B.S', \mu X.S'') \in \sequiv.
          \end{eqnarray*}
          Thus, by Eq. \ref{def:equivalence:mu2},
          \begin{equation*}
            (\send B.S', S''[X \mapsto \mu X.S'']) \in \sequiv.
          \end{equation*}
          Thus, by def.,
          \begin{equation*}
            (\send B.S, S''[X \mapsto \mu X.S'']) \in \Trans(\sequiv).
          \end{equation*}
          Thus, by Eq. \ref{def:equivalence:mu2},
          \begin{equation*}
            (t_1, t_3) = (\send B.S, \mu X.S'') \in F_\sequiv(\Trans(\sequiv)).
          \end{equation*}
      \end{enumerate}

    \item
      \label{case:transitivity:recv}
      $(t_1, t_2) = (\recv B.S, \recv B.S')$. Analogous to case
      \ref{case:transitivity:send}.

    \item
      \label{case:transitivity:echoice}
      $(t_1, t_2) = (S_1 \echoice S_2, S_1' \echoice S_2')$;
      $(S_1, S_1') \in \sequiv$;
      $(S_2, S_2') \in \sequiv$.

      Case analysis on the derivation of
      $(t_2, t_3) = (S_1' \echoice S_2', t_3) \in \sequiv$:
      \begin{enumerate}
        \item Derivation by Eq. \ref{def:equivalence:mu2} ($t_3 = \mu X.S$).
          Proceed as in case \ref{case:transitivity:send:mu2}.

        \item $t_3 = S_1'' \echoice S_2''$; $(S_1', S_1'') \in \sequiv$;
          $(S_2', S_2'') \in \sequiv$.

          By Def. \ref{def:transitive_closure},
          \begin{eqnarray*}
            && (S_1, S_1'') \in \Trans(\sequiv). \\
            && (S_2, S_2'') \in \Trans(\sequiv).
          \end{eqnarray*}
          Thus, by Eq. \ref{def:equivalence:echoice},
          \begin{equation*}
            (t_1, t_3) = (S_1 \echoice S_2, S_1'' \echoice S_2'') \in F_\sequiv(\Trans(\sequiv)).
          \end{equation*}
      \end{enumerate}

    \item
      \label{case:transitivity:ichoice}
      $(t_1, t_2) = (S_1 \ichoice S_2, S_1' \ichoice S_2')$. Analogous to
      case \ref{case:transitivity:echoice}.


    \item
      \label{case:transitivity:1}
      $(t_1, t_2) = (1, 1)$.

      Case analysis on the derivation of $(t_2, t_3) = (1, t_3) \in \sequiv$:
      \begin{enumerate}
        \item Derivation by Eq. \ref{def:equivalence:mu2} ($t_3 = \mu X.S$).
          Proceed as in case \ref{case:transitivity:send:mu2}.

        \item $t_3 = 1$. By Eq. \ref{def:equivalence:1},
          \begin{equation*}
            (t_1, t_3) = (1, 1) \in F_\sequiv(\Trans(\sequiv)).
          \end{equation*}
      \end{enumerate}

  \end{enumerate}
\end{proof}

\begin{lemma}[Variable Ordering in Environments (No Proof)]
  \label{lemma:environment_variable_ordering}
  \begin{equation*}
    XS, X, Y = XS, Y, X \quad \forall XS, X, Y.
  \end{equation*}
  %\begin{eqnarray*}
  %  &&         \forall XS, X, Y, S. \\
  %  &&         \envimplok{XS, X, Y}{S} \\
  %  &\implies& \envimplok{XS, Y, X}{S}
  %\end{eqnarray*}
\end{lemma}

%\begin{lemma}[Environment Extension (No Proof)]
  %\label{lemma:environment_extension}
  %\begin{eqnarray*}
    %&&      \forall XS, X, S. \\
    %&&      (\envimplok{XS}{S} \implies \envimplok{XS, X}{S}) \\
    %&\land& (\envimplchecked{XS}{S} \implies \envimplchecked{XS, X}{S})
  %\end{eqnarray*}
%\end{lemma}
%
%\begin{proof}
  %Induction on the construction of $S$:
  %\begin{enumerate}
    %\item
      %\label{case:environment_extension:1}
      %$S = 1$. Choose $XS, X$.
%
      %\begin{enumerate}
        %\item $(\envimplok{XS}{1} \implies \envimplok{XS, X}{1})$.
%
          %By Def. \ref{def:wellformed},
          %\begin{equation}
            %\envimplok{XS, X}{1}.
          %\end{equation}
%
        %\item $(\envimplchecked{XS}{1} \implies \envimplchecked{XS, X}{1})$.
%
          %By Def. \ref{def:wellformed},
          %\begin{equation}
            %\envimplok{XS}{1}.
          %\end{equation}
          %Thus, by Def. \ref{def:wellformed},
          %\begin{equation}
            %\envimplchecked{XS}{1}.
          %\end{equation}
      %\end{enumerate}
%
    %\item
      %\label{case:environment_extension:send}
      %$S = \send B.S'$.
%
      %Induction hypothesis:
      %\begin{eqnarray}
        %&&      \forall XS, X. \\
        %&&      (\envimplok{XS}{S'} \implies \envimplok{XS, X}{S'}) \\
        %&\land& (\envimplchecked{XS}{S'} \implies \envimplchecked{XS, X}{S'})
      %\end{eqnarray}
%
      %Choose $XS, X$.
      %\begin{enumerate}
        %\item
          %\label{case:environment_extension:send:1}
          %$\envimplok{XS}{\send B.S'} \implies \envimplok{XS, X}{\send B.S'}$.
%
          %By assumption,
          %\begin{equation}
            %\envimplok{XS}{\send B.S'}.
          %\end{equation}
          %Thus, by inversion of Def. \ref{def:wellformed},
          %\begin{equation}
            %\envimplchecked{XS}{S'}.
          %\end{equation}
          %Thus, by the induction hypothesis,
          %\begin{equation}
            %\envimplchecked{XS, X}{S'}.
          %\end{equation}
          %Thus, by Def. \ref{def:wellformed},
          %\begin{equation}
            %\envimplok{XS, X}{\send B.S'}.
          %\end{equation}
%
        %\item
          %\label{case:environment_extension:send:2}
          %$\envimplchecked{XS}{\send B.S'} \implies \envimplchecked{XS, X}{\send B.S'}$.
%
          %By assumption,
          %\begin{equation}
            %\envimplchecked{XS}{\send B.S'}.
          %\end{equation}
          %Thus, by inversion of Def. \ref{def:wellformed},
          %\begin{equation}
            %\envimplok{XS}{\send B.S'}.
          %\end{equation}
          %Thus, by case \ref{case:environment_extension:send:1},
          %\begin{equation}
            %\envimplok{XS, X}{\send B.S'}.
          %\end{equation}
          %Thus, by Def. \ref{def:wellformed},
          %\begin{equation}
            %\envimplchecked{XS, X}{\send B.S'}.
          %\end{equation}
      %\end{enumerate}
%
    %\item
      %\label{case:environment_extension:recv}
      %$S = \recv B.S'$.
%
      %Analogous to case \ref{case:environment_extension:send}.
%
    %\item
      %\label{case:environment_extension:echoice}
      %$S = S_1 \echoice S_2$.
%
      %Induction hypothesis:
      %\begin{eqnarray*}
        %&&      \forall XS, X. \\
        %&&      (\envimplok{XS}{S_1} \implies \envimplok{XS, X}{S_1}) \\
        %&\land& (\envimplok{XS}{S_2} \implies \envimplok{XS, X}{S_2}) \\
        %&\land& (\envimplchecked{XS}{S_1} \implies \envimplchecked{XS, X}{S_1}) \\
        %&\land& (\envimplchecked{XS}{S_2} \implies \envimplchecked{XS, X}{S_2})
      %\end{eqnarray*}
%
      %Choose $XS, X$.
      %\begin{enumerate}
        %\item
          %\label{case:environment_extension:echoice:1}
          %$\envimplok{XS}{S_1 \echoice S_2} \implies \envimplok{XS, X}{S_1 \echoice S_2}$.
%
          %By assumption,
          %\begin{equation}
            %\envimplok{XS}{S_1 \echoice S_2}.
          %\end{equation}
          %Thus, by inversion of Def. \ref{def:wellformed},
          %\begin{eqnarray}
            %&& \envimplchecked{XS}{S_1} \\
            %&& \envimplchecked{XS}{S_2}.
          %\end{eqnarray}
          %Thus, by the induction hypothesis,
          %\begin{eqnarray}
            %&& \envimplchecked{XS, X}{S_1} \\
            %&& \envimplchecked{XS, X}{S_2}.
          %\end{eqnarray}
          %Thus, by Def. \ref{def:wellformed},
          %\begin{eqnarray}
            %&& \envimplchecked{XS, X}{S_1} \\
            %&& \envimplchecked{XS, X}{S_2}.
          %\end{eqnarray}
          %Thus, by Def. \ref{def:wellformed},
          %\begin{equation}
            %\envimplchecked{XS, X}{S_1 \echoice S_2}.
          %\end{equation}
%
        %\item
          %\label{case:environment_extension:echoice:2}
          %$\envimplchecked{XS}{S_1 \echoice S_2} \implies \envimplchecked{XS, X}{S_1 \echoice S_2}$.
%
          %By assumption,
          %\begin{equation}
            %\envimplchecked{XS}{S_1 \echoice S_2}.
          %\end{equation}
          %Thus, by inversion of Def. \ref{def:wellformed},
          %\begin{equation}
            %\envimplok{XS}{S_1 \echoice S_2}.
          %\end{equation}
          %Thus, by case \ref{case:environment_extension:echoice:1},
          %\begin{equation}
            %\envimplok{XS, X}{S_1 \echoice S_2}.
          %\end{equation}
          %Thus, by Def. \ref{def:wellformed},
          %\begin{equation}
            %\envimplchecked{XS, X}{S_1 \echoice S_2}.
          %\end{equation}
      %\end{enumerate}
%
    %\item
      %\label{case:environment_extension:ichoice}
      %$S = S_1 \ichoice S_2$.
%
      %Analogous to case \ref{case:environment_extension:echoice}.
%
    %\item
      %\label{case:environment_extension:mu}
      %$S = \mu Y.S'$.
%
      %Induction hypothesis:
      %\begin{eqnarray*}
        %&&      \forall XS, X. \\
        %&&      (\envimplok{XS}{S'} \implies \envimplok{XS, X}{S'}) \\
        %&\land& (\envimplchecked{XS}{S'} \implies \envimplchecked{XS, X}{S'})
      %\end{eqnarray*}
%
      %Choose $XS, X$.
      %\begin{enumerate}
        %\item
          %\label{case:environment_extension:mu:1}
          %$\envimplok{XS}{\mu Y.S'} \implies \envimplok{XS, X}{\mu Y.S'}$.
%
          %By assumption,
          %\begin{equation}
            %\envimplok{XS}{\mu Y.S'}.
          %\end{equation}
          %Thus, by inversion of Def. \ref{def:wellformed},
          %\begin{equation}
            %\envimplok{XS, Y}{S'}.
          %\end{equation}
          %Thus, by the induction hypothesis,
          %\begin{equation}
            %\envimplok{XS, Y, X}{S'}.
          %\end{equation}
          %Thus, by Lemma \ref{lemma:environment_variable_ordering},
          %\begin{equation}
            %\envimplok{XS, X, Y}{S'}.
          %\end{equation}
          %Thus, by Def. \ref{def:wellformed},
          %\begin{equation}
            %\envimplok{XS, X}{\mu X.S'}.
          %\end{equation}
%
        %\item
          %\label{case:environment_extension:mu:2}
          %$\envimplchecked{XS}{\mu Y.S'} \implies \envimplchecked{XS, X}{\mu Y.S'}$.
%
          %By assumption,
          %\begin{equation}
            %\envimplchecked{XS}{\mu Y.S'}.
          %\end{equation}
          %Thus, by inversion of Def. \ref{def:wellformed},
          %\begin{equation}
            %\envimplok{XS}{\mu Y.S'}.
          %\end{equation}
          %Thus, by case \ref{case:environment_extension:mu:1},
          %\begin{equation}
            %\envimplok{XS, X}{\mu Y.S'}.
          %\end{equation}
          %Thus, by Def. \ref{def:wellformed},
          %\begin{equation}
            %\envimplchecked{XS, X}{\mu Y.S'}.
          %\end{equation}
      %\end{enumerate}
%
    %\item
      %\label{case:environment_extension:var}
      %$S = Y$. Choose $XS, X$.
%
      %\begin{enumerate}
        %\item $\envimplok{XS}{Y} \implies \envimplok{XS, X}{Y}$.
%
          %By assumption,
          %\begin{equation}
            %\envimplok{XS}{Y}.
          %\end{equation}
          %Thus, by inversion of Def. \ref{def:wellformed}, this case is
          %impossible.
%
        %\item $\envimplchecked{XS}{Y} \implies \envimplchecked{XS, X}{Y}$.
%
          %By assumption,
          %\begin{equation}
            %\envimplchecked{XS}{Y}.
          %\end{equation}
          %By inversion of Def. \ref{def:wellformed},
          %\begin{equation}
            %XS = XS', Y.
          %\end{equation}
          %By Def. \ref{def:wellformed},
          %\begin{equation}
            %\envimplchecked{XS', X, Y}{Y}.
          %\end{equation}
          %Thus, by Lemma \ref{lemma:environment_variable_ordering},
          %\begin{equation}
            %\envimplchecked{XS, X}{Y}.
          %\end{equation}
      %\end{enumerate}
  %\end{enumerate}
%\end{proof}

\begin{lemma}[Environment Compaction (No Proof)]
  \label{lemma:environment_compaction}
  \begin{equation*}
    XS, X, X = XS, X \quad \forall XS, X.
  \end{equation*}
\end{lemma}

\begin{lemma}[Overriding Exchange (No Proof)]
  \label{lemma:overriding_exchange}
  \begin{eqnarray*}
    &&         \forall f, x, y, a, b. \\
    &&         x \neq y \\
    &\implies& f[x \mapsto a][y \mapsto b] = f[y \mapsto b][x \mapsto a]
  \end{eqnarray*}
\end{lemma}

\begin{lemma}[Substitution in $L_\eta$]
  \label{lemma:substitution_trace_language}
  \begin{equation*}
    L_{\eta[X \mapsto L_\eta(T)]}(S) = L_\eta(S[X \mapsto T]) \quad \forall S, T, X, \eta.
  \end{equation*}
\end{lemma}

\begin{proof}
  Choose $T, X$. Induction over the construction of $S$.
  \begin{enumerate}
    \item
      \label{case:substitution_trace_language:1}
      $S = 1$.

      Choose $\eta$. By Def. \ref{def:trace_language},
      \begin{eqnarray*}
        &&  L_{\eta[X \mapsto L_\eta(T)]}(1) \\
        &=& \{\epsilon\}
            \quad \text{(by Eq. \ref{def:trace_language:1})} \\
        &=& L_\eta(1)
            \quad \text{(by Eq. \ref{def:trace_language:1})} \\
        &=& L_\eta(1[X \mapsto T])
            \quad \text{(by Def. of substitution)}.
      \end{eqnarray*}

    \item
      \label{case:substitution_trace_language:send}
      $S = \send B.S'$.

      Induction hypothesis:
      \begin{equation*}
        L_{\eta[X \mapsto L_\eta(T)]}(S') = L_\eta(S'[X \mapsto T]) \quad \forall \eta.
      \end{equation*}

      Choose $\eta$.
      \begin{eqnarray*}
        &&  L_{\eta[X \mapsto L_\eta(T)]}(\send B.S') \\
        &=& \{\send B\} \concat L_{\eta[X \mapsto L_\eta(T)]}(S')
            \quad \text{(by Eq. \ref{def:trace_language:send})} \\
        &=& \{\send B\} \concat L_\eta(S'[X \mapsto T])
            \quad \text{(by IH)} \\
        &=& L_\eta(\send B.(S'[X \mapsto T]))
            \quad \text{(by Eq. \ref{def:trace_language:send})} \\
        &=& L_\eta((\send B.S')[X \mapsto T])
            \quad \text{(by Def. of substitution)}.
      \end{eqnarray*}

    \item
      \label{case:substitution_trace_language:recv}
      $S = \recv B.S'$.

      Analogous to case \ref{case:substitution_trace_language:send}.

    \item
      \label{case:substitution_trace_language:echoice}
      $S = S_1 \echoice S_2$.

      Induction hypothesis:
      \begin{eqnarray*}
        &&      L_{\eta[X \mapsto L_\eta(T)]}(S_1) = L_\eta(S_1[X \mapsto T]) \\
        &\land& L_{\eta[X \mapsto L_\eta(T)]}(S_2) = L_\eta(S_2[X \mapsto T])
      \end{eqnarray*}

      Choose $\eta$.
      \begin{eqnarray*}
        &&  L_{\eta[X \mapsto L_\eta(T)]}(S_1 \echoice S_2) \\
        &=& \{\send 0\} . L_{\eta[X \mapsto L_\eta(T)]}(S_1) \union \{\send 1\} . L_{\eta[X \mapsto L_\eta(T)]}(S_2)
            \quad \text{(by Eq. \ref{def:trace_language:echoice})} \\
        &=& \{\send 0\} . L_\eta(S_1[X \mapsto T]) \union \{\send 1\} . L_\eta(S_2[X \mapsto T])
            \quad \text{(by IH)} \\
        &=& L_\eta(S_1[X \mapsto T] \echoice S_2[X \mapsto T])
            \quad \text{(by Eq. \ref{def:trace_language:echoice})} \\
        &=& L_\eta((S_1 \echoice S_2)[X \mapsto T])
            \quad \text{(by def. of substitution)}.
      \end{eqnarray*}

    \item
      \label{case:substitution_trace_language:ichoice}
      $S = S_1 \ichoice S_2$.

      Analogous to case \ref{case:substitution_trace_language:echoice}.

    \item
      \label{case:substitution_trace_language:mu1}
      $S = \mu X.S'$.

      Induction hypothesis:
      \begin{equation*}
        L_{\eta[X \mapsto L_\eta(T)]}(S') = L_\eta(S'[X \mapsto T]) \quad \forall \eta.
      \end{equation*}

      Choose $\eta$.
      \begin{eqnarray*}
        &&  L_{\eta[X \mapsto L_\eta(T)]}(\mu X.S') \\
        &=& \gfp(LX \mapsto L_{\eta[X \mapsto L_\eta(T)][X \mapsto LX]}(S'))
            \quad \text{(by Eq. \ref{def:trace_language:mu})} \\
        &=& \gfp(LX \mapsto L_{\eta[X \mapsto LX]}(S'))
            \quad \text{(by def. of overriding)} \\
        &=& L_\eta(\mu X.S')
            \quad \text{(by Eq. \ref{def:trace_language:mu})} \\
        &=& L_\eta((\mu X.S')[X \mapsto T])
            \quad \text{(by def. of substitution)}. \\
      \end{eqnarray*}

    \item
      \label{case:substitution_trace_language:mu2}
      $S = \mu Y.S'$, $X \neq Y$.

      Induction hypothesis:
      \begin{equation*}
        L_{\eta[X \mapsto L_\eta(T)]}(S') = L_\eta(S'[X \mapsto T]) \quad \forall \eta.
      \end{equation*}

      Choose $\eta$.
      \begin{eqnarray*}
        &&  L_{\eta[X \mapsto L_\eta(T)]}(\mu Y.S') \\
        &=& \gfp(LX \mapsto L_{\eta[X \mapsto L_\eta(T)][Y \mapsto LX]}(S'))
            \quad \text{(by Eq. \ref{def:trace_language:mu})} \\
        &=& \gfp(LX \mapsto L_{\eta[Y \mapsto LX][X \mapsto L_\eta(T)]}(S'))
            \quad \text{(by Lemma \ref{lemma:overriding_exchange})} \\
        &=& \gfp(LX \mapsto L_{\eta[Y \mapsto LX]}(S'[X \mapsto T]))
            \quad \text{(by IH)} \\
        &=& L_\eta((\mu Y.(S'[X \mapsto T]))
            \quad \text{(by Eq. \ref{def:trace_language:mu})} \\
        &=& L_\eta((\mu Y.S')[X \mapsto T])
            \quad \text{(by def. of substitution)}. \\
      \end{eqnarray*}

    \item
      \label{case:substitution_trace_language:var1}
      $S = X$.

      Choose $\eta$.
      \begin{eqnarray*}
        &&  L_{\eta[X \mapsto L_\eta(T)]}(X) \\
        &=& \eta[X \mapsto L_\eta(T)](X)
            \quad \text{(by Eq. \ref{def:trace_language:var})} \\
        &=& L_\eta(T)
            \quad \text{(by def. of overriding)} \\
        &=& L_\eta(X[X \mapsto T])
            \quad \text{(by def. of substitution)}
      \end{eqnarray*}

    \item
      \label{case:substitution_trace_language:var2}
      $S = Y$, $X \neq Y$.

      Choose $\eta$.
      \begin{eqnarray*}
        &&  L_{\eta[X \mapsto L_\eta(T)]}(Y) \\
        &=& \eta[X \mapsto L_\eta(T)](Y)
            \quad \text{(by Eq. \ref{def:trace_language:var})} \\
        &=& \eta(Y)
            \quad \text{(by def. of overriding)} \\
        &=& L_\eta(Y)
            \quad \text{(by Eq. \ref{def:trace_language:var})} \\
        &=& L_\eta(Y[X \mapsto T])
            \quad \text{(by def. of substitution)}
      \end{eqnarray*}

  \end{enumerate}
\end{proof}

\begin{lemma}[Trace Language of $\mu$ Expansion]
  \label{lemma:mu_expansion_trace_language}
  \begin{equation*}
    L_\eta(\mu X.S) = L_\eta(S[X \mapsto \mu X.S]) \quad \forall \eta, X, S.
  \end{equation*}
\end{lemma}

\begin{proof}
  Choose $\eta, X, S$.
  \begin{eqnarray*}
    &&  L_\eta(\mu X.S) \\
    &=& \gfp(LX \mapsto L_{\eta[X \mapsto LX]}(S))
        \quad \text{(by Eq. \ref{def:trace_language:mu})} \\
    &=& L_{\eta[X \mapsto L_\eta(\mu X.S)]}(S)
        \quad \text{(due to $L_\eta(\mu X.S)$ a fixed point)} \\
    &=& L_\eta(S[X \mapsto \mu X.S])
        \quad \text{(by Lemma \ref{lemma:substitution_trace_language})}.
  \end{eqnarray*}
\end{proof}

\begin{lemma}[Wellformedness-Preserving Substitution]
  \label{lemma:substitution_wellformedness}
  \begin{eqnarray*}
    &&         \forall S, T, XS, X. \\
    &&         \envimplok{XS, X}{S} \\
    &\land&    \envimplok{XS}{T} \\
    &\implies& \envimplok{XS}{S[X \mapsto T]}.
  \end{eqnarray*}
\end{lemma}

\begin{proof}
  Choose $X, T$. Induction over the construction of $S$.
  \begin{enumerate}
    \item
      \label{case:substitution_wellformedness:1}
      $S = 1$.

      Choose $XS$. By Def. \ref{def:wellformed},
      \begin{equation*}
        \envimplok{XS}{1}.
      \end{equation*}

    \item
      \label{case:substitution_wellformedness:send}
      $S = \send B.S'$.

      Induction hypothesis:
      \begin{eqnarray*}
        &&         \forall XS. \\
        &&         \envimplok{XS, X}{S'} \\
        &\land&    \envimplok{XS}{T} \\
        &\implies& \envimplok{XS}{S'[X \mapsto T]}.
      \end{eqnarray*}

      Choose $XS$. By assumption,
      \begin{equation*}
        \envimplok{XS, X}{\send B.S'}.
      \end{equation*}
      Thus, by inversion of Def. \ref{def:wellformed},
      \begin{equation*}
        \envimplchecked{XS, X}{S'}.
      \end{equation*}
      Case analysis on the derivation of $(\envimplchecked{XS, X}{S'})$:
      \begin{enumerate}
        \item $S' = X$.
          By assumption,
          \begin{equation*}
            \envimplok{XS}{T}.
          \end{equation*}
          Thus, by the def. of substitution,
          \begin{equation*}
            \envimplok{XS}{S'[X \mapsto T]}.
          \end{equation*}

        \item $\envimplok{XS, X}{S'}$.
          By the induction hypothesis,
          \begin{equation*}
            \envimplok{XS}{S'[X \mapsto T]}.
          \end{equation*}
      \end{enumerate}

      Thus, in both cases, by Def. \ref{def:wellformed},
      \begin{equation*}
        \envimplchecked{XS}{S'[X \mapsto T]}.
      \end{equation*}
      Thus, by Def. \ref{def:wellformed},
      \begin{equation*}
        \envimplok{XS}{!B.(S'[X \mapsto T])}.
      \end{equation*}
      Thus, by the def. of substitution,
      \begin{equation*}
        \envimplok{XS}{(!B.S')[X \mapsto T]}.
      \end{equation*}

    \item
      \label{case:substitution_wellformedness:recv}
      $S = \recv B.S'$.

      Analogous to case \ref{case:substitution_wellformedness:send}.

    \item
      \label{case:substitution_wellformedness:echoice}
      $S = S_1 \echoice S_2$.

      Induction hypothesis: For $i \in \{1, 2\}$,
      \begin{eqnarray*}
        &&         \forall XS. \\
        &&         \envimplok{XS, X}{S_i} \\
        &\land&    \envimplok{XS}{T} \\
        &\implies& \envimplok{XS}{S_i[X \mapsto T]};
      \end{eqnarray*}

      Choose $XS$. By assumption,
      \begin{equation*}
        \envimplok{XS, X}{S_1 \echoice S_2}.
      \end{equation*}
      By inversion of Def. \ref{def:wellformed},
      \begin{eqnarray*}
        &&      \envimplchecked{XS, X}{S_1} \\
        &\land& \envimplchecked{XS, X}{S_2}.
      \end{eqnarray*}
      Case analysis on the derivation of $\envimplchecked{XS, X}{S_i}$:
      \begin{enumerate}
        \item $S_i = X$.
          By assumption,
          \begin{equation*}
            \envimplok{XS}{T}.
          \end{equation*}
          Thus, by the def. of substitution,
          \begin{equation*}
            \envimplok{XS}{S_i[X \mapsto T]}.
          \end{equation*}

        \item $\envimplok{XS, X}{S_i}$.
          Thus, the induction hypothesis,
          \begin{equation*}
            \envimplok{XS}{S_i[X \mapsto T]}.
          \end{equation*}
      \end{enumerate}

      Thus, in both cases, by Def. \ref{def:wellformed},
      \begin{equation*}
        \envimplchecked{XS}{S_i[X \mapsto T]}.
      \end{equation*}
      Thus, by Def. \ref{def:wellformed},
      \begin{equation*}
        \envimplok{XS}{S_1[X \mapsto T] \echoice S_2[X \mapsto T]}.
      \end{equation*}
      Thus, by the def. of substitution,
      \begin{equation*}
        \envimplok{XS}{(S_1 \echoice S_2)[X \mapsto T]}.
      \end{equation*}

    \item
      \label{case:substitution_wellformedness:ichoice}
      $S = S_1 \ichoice S_2$.

      Analogous to case \ref{case:substitution_wellformedness:echoice}.

    \item
      \label{case:substitution_wellformedness:mu1}
      $S = \mu X.S'$.

      Choose $XS$. By assumption,
      \begin{equation*}
        \envimplok{XS, X}{\mu X.S'}.
      \end{equation*}
      Thus, by inversion of Def. \ref{def:wellformed},
      \begin{equation*}
        \envimplok{XS, X, X}{S'}.
      \end{equation*}
      Thus, by Lemma \ref{lemma:environment_compaction},
      \begin{equation*}
        \envimplok{XS, X}{S'}.
      \end{equation*}
      Thus, by Def. \ref{def:wellformed},
      \begin{equation*}
        \envimplok{XS}{\mu X.S'}.
      \end{equation*}
      Thus, by the def. of substitution,
      \begin{equation*}
        \envimplok{XS}{(\mu X.S')[X \mapsto T]}.
      \end{equation*}

    \item
      \label{case:substitution_wellformedness:mu2}
      $S = \mu Y.S'$, $X \neq Y$.

      Induction hypothesis:
      \begin{eqnarray*}
        &&         \forall XS. \\
        &&         \envimplok{XS, X}{S'} \\
        &\land&    \envimplok{XS}{T} \\
        &\implies& \envimplok{XS}{S'[X \mapsto T]};
      \end{eqnarray*}

      Choose $XS$. By assumption,
      \begin{equation*}
        \envimplok{XS, X}{\mu Y.S'}.
      \end{equation*}
      Thus, by inversion of Def. \ref{def:wellformed},
      \begin{equation*}
        \envimplok{XS, X, Y}{S'}.
      \end{equation*}
      Thus, by Lemma \ref{lemma:environment_variable_ordering},
      \begin{equation*}
        \envimplok{XS, Y, X}{S'}.
      \end{equation*}
      Thus, by the induction hypothesis,
      \begin{equation*}
        \envimplok{XS, Y}{S'[X \mapsto T]}.
      \end{equation*}
      Thus, by Def. \ref{def:wellformed},
      \begin{equation*}
        \envimplok{XS}{\mu Y.(S'[X \mapsto T])}.
      \end{equation*}
      Thus, by the def. of substitution,
      \begin{equation*}
        \envimplok{XS}{(\mu Y.S')[X \mapsto T])}.
      \end{equation*}

  \end{enumerate}
\end{proof}

\begin{lemma}[Inversion of Wellformedness]
  \label{lemma:inversion_wellformedness}
  \begin{eqnarray*}
    &&      \forall S, S_1, S_2, X. \\
    &&      (\wf{\send B.S} \implies \wf{S}) \\
    &\land& (\wf{\recv B.S} \implies \wf{S}) \\
    &\land& (\wf{S_1 \echoice S_2} \implies \wf{S_1} \land \wf{S_2}) \\
    &\land& (\wf{S_1 \ichoice S_2} \implies \wf{S_1} \land \wf{S_2}) \\
    &\land& (\wf{\mu X.S} \implies \wf{S[X \mapsto \mu X.S]})
  \end{eqnarray*}
\end{lemma}

\begin{proof}
  Choose $S, S_1, S_2, X$.
  \begin{enumerate}
    \item
      \label{case:inversion_wellformedness:send}
      $\wf{\send B.S} \implies \wf{S}$.

      By assumption,
      \begin{equation*}
        \envimplok{0}{\send B.S}.
      \end{equation*}
      Thus, by inversion of Def. \ref{def:wellformed},
      \begin{equation*}
        \envimplchecked{0}{S}.
      \end{equation*}
      Thus, by inversion of Def. \ref{def:wellformed},
      \begin{equation*}
        \envimplok{0}{S}.
      \end{equation*}

    \item
      \label{case:inversion_wellformedness:recv}
      $\wf{\recv B.S} \implies \wf{S}$.

      Analogous to case \ref{case:inversion_wellformedness:send}.

    \item
      \label{case:inversion_wellformedness:echoice}
      $\wf{S_1 \echoice S_2} \implies \wf{S_1} \land \wf{S_2}$.

      By assumption,
      \begin{equation*}
        \envimplok{0}{S_1 \echoice S_2}.
      \end{equation*}
      Thus, by inversion of Def. \ref{def:wellformed},
      \begin{eqnarray*}
        &&      \envimplchecked{0}{S_1} \\
        &\land& \envimplchecked{0}{S_2}.
      \end{eqnarray*}
      Thus, by inversion of Def. \ref{def:wellformed},
      \begin{eqnarray*}
        &&      \envimplok{0}{S_1} \\
        &\land& \envimplok{0}{S_2}.
      \end{eqnarray*}

    \item
      \label{case:inversion_wellformedness:ichoice}
      $\wf{S_1 \ichoice S_2} \implies \wf{S_1} \land \wf{S_2}$.

      Analogous to case \ref{case:inversion_wellformedness:echoice}.

    \item
      \label{case:inversion_wellformedness:mu}
      $\wf{\mu X.S} \implies \wf{S[X \mapsto \mu X.S]}$.

      By assumption,
      \begin{equation*}
        \envimplok{0}{\mu X.S}.
      \end{equation*}
      By inversion of Def. \ref{def:wellformed},
      \begin{equation*}
        \envimplok{0, X}{S}.
      \end{equation*}
      By Lemma \ref{lemma:substitution_wellformedness},
      \begin{equation*}
        \envimplok{0}{S[X \mapsto \mu X.S]}.
      \end{equation*}
  \end{enumerate}
\end{proof}

\begin{theorem}[Trace Equality implies Equivalence]
  \label{th:equality_equivalence}
  \begin{eqnarray*}
    &&         \forall S, S' \in \Sy. \\
    &&         \wf{S} \\
    &\land&    \wf{S'} \\
    &\land&    (\forall \eta.\, L_\eta(S) = L_\eta(S')) \\
    &\implies& (S, S') \in \sequiv.
  \end{eqnarray*}
\end{theorem}

\begin{proof}
  Let
  \begin{equation*}
    R := \{(S, S') \| \wf{S}, \wf{S'}, \forall \eta.\, L_\eta(S) = L_\eta(S') \}.
  \end{equation*}
  We will show that
  \begin{equation*}
    R \subseteq F_\sequiv(R)
  \end{equation*}
  which, by the fixed-point theorem, implies
  \begin{equation*}
    R \subseteq \sequiv.
  \end{equation*}

  Choose $(S, S') \in R$.
  Case analysis on the construction of $S$ and $S'$:
  \begin{enumerate}
    \item
      \label{case:equality_equivalence:mu1}
      $(S, S') = (\mu X.T, S')$.

      By Lemma \ref{lemma:mu_expansion_trace_language},
      \begin{equation*}
        \forall \eta.\, L_\eta(T[X \mapsto \mu X.T]) = L_\eta(\mu X.T).
      \end{equation*}
      Thus, by assumption,
      \begin{equation}
        \label{eq:equality_equivalence:1:mu1:ls}
        \forall \eta.\, L_\eta(T[X \mapsto \mu X.T]) = L_\eta(S').
      \end{equation}

      By assumption,
      \begin{equation*}
        \wf{\mu X.T}.
      \end{equation*}
      Thus, by Lemma \ref{lemma:inversion_wellformedness},
      \begin{equation}
        \label{eq:equality_equivalence:1:mu1:wf}
        \wf{T[X \mapsto \mu X.T]}.
      \end{equation}

      Due to Eqs. \ref{eq:equality_equivalence:1:mu1:ls} and
      \ref{eq:equality_equivalence:1:mu1:wf} and by assumption
      $\wf{S'}$,
      \begin{equation*}
        (T[X \mapsto \mu X.T], S') \in R.
      \end{equation*}
      Thus, by Eq. \ref{def:equivalence:mu1},
      \begin{equation*}
        (S, S') = (\mu X.T, S') \in F_\sequiv(R).
      \end{equation*}

    \item
      \label{case:equality_equivalence:mu2}
      $(S, S') = (S, \mu X.T)$.

      Symmetric to case \ref{case:equality_equivalence:mu1}.

    \item
      \label{case:equality_equivalence:send}
      $(S, S') = (\send B.T, \send B.T')$.

      By Def. \ref{def:trace_language},
      \begin{eqnarray*}
        &&      \forall \eta. \\
        &&      L_\eta(\send B.T)  = \{\send B\} \concat L_\eta(T) \\
        &\land& L_\eta(\send B.T') = \{\send B\} \concat L_\eta(T')
      \end{eqnarray*}
      Thus, by inversion of $L_\eta(S) = L_\eta(S')$:
      \begin{equation}
        \label{eq:equality_equivalence:send:ls}
        \forall \eta.\, L_\eta(T) = L_\eta(T').
      \end{equation}

      By Lemma \ref{lemma:inversion_wellformedness},
      \begin{eqnarray}
        \label{eq:equality_equivalence:send:wf:1} &&      \wf{T} \\
        \label{eq:equality_equivalence:send:wf:2} &\land& \wf{T'}
      \end{eqnarray}

      Due to Eqs. \ref{eq:equality_equivalence:send:ls},
      \ref{eq:equality_equivalence:send:wf:1} and
      \ref{eq:equality_equivalence:send:wf:2},
      \begin{equation*}
        (T, T') \in R.
      \end{equation*}
      Thus, by Eq. \ref{def:equivalence:send},
      \begin{equation*}
        (\send B.T, \send B.T') \in F_\sequiv(R).
      \end{equation*}

    \item
      \label{case:equality_equivalence:recv}
      $(S, S') = (\recv B.T, \recv B.T')$.

      Analogous to case \ref{case:equality_equivalence:send}.

    \item
      \label{case:equality_equivalence:echoice}
      $(S, S') = (T_1 \echoice T_2, T_1' \echoice T_2')$.

      By Def. \ref{def:trace_language},
      \begin{eqnarray*}
        &&      \forall \eta. \\
        &&      L_\eta(T_1 \echoice T_2)   = \{\send 0\} \concat L_\eta(T_1) \union \{\send 1\} \concat L_\eta(T_2) \\
        &\land& L_\eta(T_1' \echoice T_2') = \{\send 0\} \concat L_\eta(T_1') \union \{\send 1\} \concat L_\eta(T_2') \\
      \end{eqnarray*}
      Thus, by inversion of $L_\eta(S) = L_\eta(S')$:
      \begin{eqnarray}
        \label{eq:equality_equivalence:echoice:ls:1} && \forall \eta.\, L_\eta(T_1) = L_\eta(T_1') \\
        \label{eq:equality_equivalence:echoice:ls:2} && \forall \eta.\, L_\eta(T_2) = L_\eta(T_2').
      \end{eqnarray}

      By Lemma \ref{lemma:inversion_wellformedness},
      \begin{eqnarray}
        \label{eq:equality_equivalence:echoice:wf:1} &&      \wf{T_1} \\
        \label{eq:equality_equivalence:echoice:wf:2} &\land& \wf{T_2} \\
        \label{eq:equality_equivalence:echoice:wf:3} &\land& \wf{T_1'} \\
        \label{eq:equality_equivalence:echoice:wf:4} &\land& \wf{T_2'}
      \end{eqnarray}

      Due to Eqs.
      \ref{eq:equality_equivalence:echoice:ls:1},
      \ref{eq:equality_equivalence:echoice:ls:2},
      \ref{eq:equality_equivalence:echoice:wf:1},
      \ref{eq:equality_equivalence:echoice:wf:2},
      \ref{eq:equality_equivalence:echoice:wf:3} and
      \ref{eq:equality_equivalence:echoice:wf:4},
      \begin{eqnarray*}
        &&      (T_1, T_1') \in R \\
        &\land& (T_2, T_2') \in R.
      \end{eqnarray*}
      Thus, by Eq. \ref{def:equivalence:echoice},
      \begin{equation*}
        (T_1 \echoice T_2, T_1' \echoice T_2') \in F_\sequiv(R).
      \end{equation*}

    \item
      \label{case:equality_equivalence:ichoice}
      $(S, S') = (T_1 \ichoice T_2, T_1' \ichoice T_2')$.

      Analogous to case \ref{case:equality_equivalence:echoice}.

    \item
      \label{case:equality_equivalence:var1}
      $(S, S') = (X, S')$.

      By assumption,
      \begin{equation*}
        \wf{X}.
      \end{equation*}
      But by inversion of Def. \ref{def:wellformed}, this is impossible.

    \item
      \label{case:equality_equivalence:var2}
      $(S, S') = (S, X)$.

      Symmetric to case \ref{case:equality_equivalence:var2}.

    \item
      \label{case:equality_equivalence:other}
      All other combinations of $(S, S')$ are impossible due to the assumption
      that $L_\eta(S) = L_\eta(S')$.
  \end{enumerate}
\end{proof}
\end{document}
